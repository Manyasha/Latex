\documentclass[10pt]{article}

\usepackage[UTF8]{inputenc}
\usepackage[T2A]{fontenc}
\usepackage[english,russian]{babel}
\usepackage{amssymb,latexsym,amsmath,amscd}
\usepackage[left=3cm,right=1.5cm,top=2cm,bottom=2cm,bindingoffset=0cm]{geometry}
\usepackage{indentfirst}
\pagestyle{empty}
\renewcommand{\thesection}{\arabic{section}.}
\renewcommand{\thesubsection}{\arabic{section}.\arabic{subsection}.}
\author{}
\date{}
\usepackage{ragged2e}
\justifying
\parindent=1.5cm

\begin{document}
\title{\bf Сложность}
\maketitle
\thispagestyle{empty}
\begin{flushright}
\begin{minipage}{0.47\hsize}
\small
Врач, строитель и программистка спорили о том,\linebreak чья профессия древнее. Врач заметил: "В Библии\linebreak сказано, что Бог сотворил Еву из ребра Адама.\linebreak Такая операция может быть проведена только\linebreak хирургом, поэтому я по праву могу утверждать,\linebreak что моя профессия самая древняя в мире". Тут\linebreak вмешался строитель и сказал: "Но еще раньше в\linebreak Книге Бытия сказано, что Бог сотворил из хаоса\linebreak небо и землю. Это было первое и, несомненно,\linebreak наиболее выдающееся строительство. Поэтому,\linebreak дорогой доктор, вы не правы. Моя профессия\linebreak самая древняя в мире". Программистка при этих\linebreak словах откинулась в кресле и с улыбкой\linebreak произнесла: "А кто же по-вашему сотворил\linebreak хаос?"
\end{minipage}
\end{flushright}

\section{Сложность, присущая программному обеспечению}
\subsection{Простые и сложные программные системы}
Звезда в преддверии коллапса; ребенок, который учится читать; клетки крови, атакующие\linebreak вирус, - это только некоторые из потрясающе сложных объектов физического мира.\linebreak  Компьютерные программы тоже бывают сложными, однако их сложность совершенно другого\linebreak  рода. Брукс пишет: "Эйнштейн утверждал, что должны существовать простые объяснения\linebreak  природных процессов, так как Бог не действует из каприза или по произволу. У программиста нет\linebreak  такого утешения: сложность, с которой он должен справиться, лежит в самой природе системы"\linebreak  [1].

Мы знаем, что не все программные системы сложны. Существует множество программ,\linebreak  которые задумываются, разрабатываются, сопровождаются и используются одним и тем же\linebreak  человеком. Обычно это начинающий программист или профессионал, работающий изолированно.\linebreak  Мы не хотим сказать, что все такие системы плохо сделаны или, тем более, усомниться в\linebreak  квалификации их создателей. Но такие системы, как правило, имеют очень ограниченную область\linebreak  применения и короткое время жизни. Обычно их лучше заменить новыми, чем пытаться повторно\linebreak  использовать, переделывать или расширять. Разработка подобных программ скорее утомительна,\linebreak  чем сложна, так что изучение этого процесса нас не интересует. 

Нас интересует разработка того, что мы будем называть промышленными программными\linebreak  продуктами. Они применяются для решения самых разных задач, таких, например, как системы с\linebreak  обратной связью, которые управляют или сами управляются событиями физического мира и для\linebreak  которых ресурсы времени и памяти ограничены; задачи поддержания целостности информации\linebreak  объемом в сотни тысяч записей при параллельном доступе к ней с обновлениями и запросами;\linebreak  системы управления и контроля за реальными процессами (например, диспетчеризация\linebreak  воздушного или железнодорожного транспорта). Системы подобного типа обычно имеют большое\linebreak  время жизни, и большое количество пользователей оказывается в зависимости от их нормального\linebreak  функционирования. В мире промышленных программ мы также встречаем среды разработки,\linebreak  которые упрощают создание приложений в конкретных областях, и программы, которые\linebreak  имитируют определенные стороны человеческого интеллекта. 

Существенная черта промышленной программы - уровень сложности: один разработчик\linebreak  практически не в состоянии охватить все аспекты такой системы. Грубо говоря, сложность\linebreak  промышленных программ превышает возможности человеческого интеллекта. Увы, но сложность,\linebreak  о которой мы говорим, по-видимому, присуща всем большим программных системам. Говоря\linebreak  "присуща", мы имеем в виду, что эта сложность здесь неизбежна: с ней можно справиться, но\linebreak  избавиться от нее нельзя. 

Конечно, среди нас всегда есть гении, которые в одиночку могут выполнить работу группы\linebreak  обычных людей-разработчиков и добиться в своей области успеха, сравнимого с достижениями\linebreak  Франка Ллойда Райта или Леонардо да Винчи. Такие люди нам нужны как архитекторы, которые\linebreak  изобретают новые идиомы, механизмы и основные идеи, используемые затем при разработке\linebreak  других систем. Однако, как замечает Петерс: "В мире очень мало гениев, и не надо думать, будто в\linebreak  среде программистов их доля выше средней" [2]. Несмотря на то, что все мы чуточку гениальны, в\linebreak  промышленном программировании нельзя постоянно полагаться на божественное вдохновение,\linebreak  которое обязательно поможет нам. Поэтому мы должны рассмотреть более надежные способы\linebreak  конструирования сложных систем. Для лучшего понимания того, чем мы собираемся управлять,\linebreak  сначала ответим на вопрос: почему сложность присуща всем большим программным системам? 
\subsection{Почему программному обеспечению присуща сложность?}
Как говорит Брукс, "сложность программного обеспечения - отнюдь не случайное его свойство" [3]. Сложность вызывается четырьмя основными причинами: 
•	сложностью реальной предметной области, из которой исходит заказ на разработку;
•	трудностью управления процессом разработки;
•	необходимостью обеспечить достаточную гибкость программы;
•	неудовлетворительными способами описания поведения больших дискретных систем.

Сложность реального мира. Проблемы, которые мы пытаемся решить с помощью\linebreak  программного обеспечения, часто неизбежно содержат сложные элементы, а к соответствующим\linebreak  программам предъявляется множество различных, порой взаимоисключающих требований.\linebreak  Рассмотрим необходимые характеристики электронной системы многомоторного самолета,\linebreak  сотовой телефонной коммутаторной системы и робота. Достаточно трудно понять, даже в общих\linebreak  чертах, как работает каждая такая система. Теперь прибавьте к этому дополнительные требования\linebreak  (часто не формулируемые явно), такие как удобство, производительность, стоимость,\linebreak  выживаемость и надежность! Сложность задачи и порождает ту сложность программного\linebreak  продукта, о которой пишет Брукс. 

Эта внешняя сложность обычно возникает из-за "нестыковки" между пользователями\linebreak  системы и ее разработчиками: пользователи с трудом могут объяснить в форме, понятной\linebreak  разработчикам, что на самом деле нужно сделать. Бывают случаи, когда пользователь лишь\linebreak  смутно представляет, что ему нужно от будущей программной системы. Это в основном\linebreak  происходит не из-за ошибок с той или иной стороны; просто каждая из групп специализируется в\linebreak  своей области, и ей недостает знаний партнера. У пользователей и разработчиков разные взгляды\linebreak  на сущность проблемы, и они делают различные выводы о возможных путях ее решения. На\linebreak  самом деле, даже если пользователь точно знает, что ему нужно, мы с трудом можем однозначно\linebreak  зафиксировать все его требования. Обычно они отражены на многих страницах текста,\linebreak  "разбавленных" немногими рисунками. Такие документы трудно поддаются пониманию, они\linebreak  открыты для различных интерпретаций и часто содержат элементы, относящиеся скорее к\linebreak  дизайну, чем к необходимым требованиям разработки.
 
Дополнительные сложности возникают в результате изменений требований к программной\linebreak  системе уже в процессе разработки. В основном требования корректируются из-за того, что само\linebreak  осуществление программного проекта часто изменяет проблему. Рассмотрение первых\linebreak  результатов - схем, прототипов, - и использование системы после того, как она разработана и\linebreak  установлена, заставляют пользователей лучше понять и отчетливей сформулировать то, что им\linebreak  действительно нужно. В то же время этот процесс повышает квалификацию разработчиков в\linebreak  предметной области и позволяет им задавать более осмысленные вопросы, которые проясняют\linebreak  темные места в проектируемой системе. 

Большая программная система - это крупное капиталовложение, и мы не можем позволить\linebreak  себе выкидывать сделанное при каждом изменении внешних требований. Тем не менее даже\linebreak  большие системы имеют тенденцию к эволюции в процессе их использования: следовательно,\linebreak  встает задача о том, что часто неправильно называют сопровождением программного обеспечения.\linebreak  Чтобы быть более точными, введем несколько терминов: 
•	под сопровождением понимается устранение ошибок;
•	под эволюцией - внесение изменений в систему в ответ на изменившиеся требования к ней;
•	под сохранением - использование всех возможных и невозможных способов для поддержания жизни в дряхлой и распадающейся на части системе.

К сожалению, опыт показывает, что существенный процент затрат на разработку программных систем тратится именно на сохранение. 

Трудности управления процессом разработки. Основная задача разработчиков состоит в\linebreak  создании иллюзии простоты, в защите пользователей от сложности описываемого предмета или\linebreak  процесса. Размер исходных текстов программной системы отнюдь не входит в число ее главных\linebreak  достоинств, поэтому мы стараемся делать исходные тексты более компактными, изобретая\linebreak  хитроумные и мощные методы, а также используя среды разработки уже существующих проектов\linebreak  и программ. Однако новые требования для каждой новой системы неизбежны, а они приводят к\linebreak  необходимости либо создавать много программ "с нуля", либо пытаться по-новому использовать\linebreak  существующие. Всего 20 лет назад программы объемом в несколько тысяч строк на ассемблере\linebreak  выходили за пределы наших возможностей. Сегодня обычными стали программные системы,\linebreak  размер которых исчисляется десятками тысяч или даже миллионами строк на языках высокого\linebreak  уровня. Ни один человек никогда не сможет полностью понять такую систему. Даже если мы\linebreak  правильно разложим ее на составные части, мы все равно получим сотни, а иногда и тысячи\linebreak  отдельных модулей. Поэтому такой объем работ потребует привлечения команды разработчиков,\linebreak  в идеале как можно меньшей по численности. Но какой бы она ни была, всегда будут возникать\linebreak  значительные трудности, связанные с организацией коллективной разработки. Чем больше\linebreak  разработчиков, тем сложнее связи между ними и тем сложнее координация, особенно если\linebreak  участники работ географически удалены друг от друга, что типично в случае очень больших\linebreak  проектов. Таким образом, при коллективном выполнении проекта главной задачей руководства\linebreak  является поддержание единства и целостности разработки. \linebreak


Гибкость программного обеспечения. Домостроительная компания обычно не имеет\linebreak  собственного лесхоза, который бы ей поставлял лес для пиломатериалов; совершенно необычно,\linebreak  чтобы монтажная фирма соорудила свой завод для изготовления стальных балок под будущее\linebreak здание. Однако в программной индустрии такая практика - дело обычное. Программирование\linebreak обладает предельной гибкостью, и разработчик может сам обеспечить себя всеми необходимыми\linebreak элементами, относящимися к любому уровню абстракции. Такая гибкость чрезвычайно\linebreak соблазнительна. Она заставляет разработчика создавать своими силами все базовые строительные\linebreak блоки будущей конструкции, из которых составляются элементы более высоких уровней\linebreak абстракции. В отличие от строительной индустрии, где существуют единые стандарты на многие\linebreak конструктивные элементы и качество материалов, в программной индустрии таких стандартов\linebreak почти нет. Поэтому программные разработки остаются очень трудоемким делом.
 
Проблема описания поведения больших дискретных систем. Когда мы кидаем вверх\linebreak мяч, мы можем достоверно предсказать его траекторию, потому что знаем, что в нормальных\linebreak условиях здесь действуют известные физические законы. Мы бы очень удивились, если бы, кинув\linebreak мяч с чуть большей скоростью, увидели, что он на середине пути неожиданно остановился и резко\linebreak изменил направление движения [Даже простые непрерывные системы могут иметь сложное\linebreak поведение ввиду наличия хаоса. Хаос привносит случайность, исключающую точное предсказание\linebreak будущего состояния системы. Например, зная начальное положение двух капель воды в потоке,\linebreak мы не можем точно предсказать, на каком расстоянии друг от друга они окажутся по прошествии\linebreak некоторого времени. Хаос проявляется в таких различных системах, как атмосферные процессы,\linebreak химические реакции, биологические системы и даже компьютерные сети. К счастью, скрытый\linebreak порядок, по-видимому, есть во всех хаотических системах, в виде так называемых аттракторов]\linebreak. В недостаточно отлаженной программе моделирования полета мяча такая ситуация легко может возникнуть. 

\subsection{Последствия неограниченной сложности}
"Чем сложнее система, тем легче ее полностью развалить" [5]. Строитель едва ли\linebreak согласится расширить фундамент уже построенного 100-этажного здания. Это не просто дорого:\linebreak делать такие вещи значит напрашиваться на неприятности. Но что удивительно, пользователи\linebreak программных систем, не задумываясь, ставят подобные задачи перед разработчиками. Это,\linebreak утверждают они, всего лишь технический вопрос для программистов. 

Наше неумение создавать сложные программные системы проявляется в проектах,\linebreak которые выходят за рамки установленных сроков и бюджетов и к тому же не соответствуют\linebreak начальным требованиям. Мы часто называем это кризисом программного обеспечения, но, честно\linebreak говоря, недомогание, которое тянется так долго, становится нормой. К сожалению, этот кризис\linebreak приводит к разбазариванию человеческих ресурсов - самого драгоценного товара - и к\linebreak существенному ограничению возможностей создания новых продуктов. Сейчас просто не хватает\linebreak хороших программистов, чтобы обеспечить всех пользователей нужными программами. Более\linebreak того, существенный процент персонала, занятого разработками, в любой организации часто\linebreak должен заниматься сопровождением и сохранением устаревших программ. С учетом прямого и\linebreak косвенного вклада индустрии программного обеспечения в развитие экономики большинства\linebreak ведущих стран, нельзя позволить, чтобы существующая ситуация осталась без изменений. 

Как мы можем изменить положение дел? Так как проблема возникает в результате\linebreak сложности структуры программных продуктов, мы предлагаем сначала рассмотреть способы\linebreak работы со сложными структурами в других областях. В самом деле, можно привести множество\linebreak примеров успешно функционирующих сложных систем. Некоторые из них созданы человеком,\linebreak например: космический челнок Space Shuttle, туннель под Ла-Маншем, большие фирмы типа\linebreak Microsoft или General Electric. В природе существуют еще более сложные системы, например\linebreak система кровообращения у человека или растение. 

\section{Структура сложных систем}
\subsection{Примеры сложных систем} 
Структура персонального компьютера. Персональный компьютер (ПК) - прибор\linebreak умеренной сложности. Большинство ПК состоит из одних и тех же основных элементов:\linebreak системной платы, монитора, клавиатуры и устройства внешней памяти какого-либо типа (гибкого\linebreak или жесткого диска). Мы можем взять любую из этих частей и разложить ее в свою очередь на\linebreak составляющие. Системная плата, например, содержит оперативную память, центральный\linebreak процессор (ЦП) и шину, к которой подключены периферийные устройства. Каждую из этих частей\linebreak можно также разложить на составляющие: ЦП состоит из регистров и схем управления, которые\linebreak сами состоят из еще более простых деталей: диодов, транзисторов и т.д. 

Это пример сложной иерархической системы. Персональный компьютер нормально\linebreak работает благодаря четкому совместному функционированию всех его составных частей. Вместе\linebreak эти части образуют логическое целое. Мы можем понять, как работает компьютер, только потому,\linebreak что можем рассматривать отдельно каждую его составляющую. Таким образом, можно изучать\linebreak устройства монитора и жесткого диска независимо друг от друга. Аналогично можно изучать\linebreak арифметическую часть ЦП, не рассматривая при этом подсистему памяти. 

Дело не только в том, что сложная система ПК иерархична, но в том, что уровни этой\linebreak иерархии представляют различные уровни абстракции, причем один надстроен над другим и\linebreak каждый может быть рассмотрен (понят) отдельно. На каждом уровне абстракции мы находим\linebreak набор устройств, которые совместно обеспечивают некоторые функции более высокого уровня, и\linebreak выбираем уровень абстракции, исходя из наших специфических потребностей. Например, пытаясь\linebreak исследовать проблему синхронизации обращений к памяти, можно оставаться на уровне\linebreak логических элементов компьютера, но этот уровень абстракции не подходит при поиске ошибки в\linebreak прикладной программе, работающей с электронными таблицами. 

Структура растений и животных. Ботаник пытается понять сходство и различия\linebreak растений, изучая их морфологию, то есть форму и структуру. Растения - это сложные\linebreak многоклеточные организмы. В результате совместной деятельности различных органов растений\linebreak происходят такие сложные типы поведения, как фотосинтез и всасывание влаги. 

Растение состоит из трех основных частей: корни, стебли и листья. Каждая из них имеет\linebreak свою особую структуру. Корень, например, состоит из корневых отростков, корневых волосков,\linebreak верхушки корня и т.д. Рассматривая срез листа, мы видим его эпидермис, мезофилл и сосудистую\linebreak ткань. Каждая из этих структур, в свою очередь, представляет собой набор клеток. Внутри каждой\linebreak клетки можно выделить следующий уровень, который включает хлоропласт, ядро и т.д. Так же,\linebreak как у компьютера, части растения образуют иерархию, каждый уровень которой обладает\linebreak собственной независимой сложностью. 

Все части на одном уровне абстракции взаимодействуют вполне определенным образом.\linebreak Например, на высшем уровне абстракции, корни отвечают за поглощение из почвы воды и\linebreak минеральных веществ. Корни взаимодействуют со стеблями, которые передают эти вещества\linebreak листьям. Листья в свою очередь используют воду и минеральные вещества, доставляемые\linebreak стеблями, и производят при помощи фотосинтеза необходимые элементы. 

Для каждого уровня абстракции всегда четко разграничено "внешнее" и "внутреннее".\linebreak Например, можно установить, что части листа совместно обеспечивают функционирование листа\linebreak в целом и очень слабо взаимодействуют или вообще прямо не взаимодействуют с элементами\linebreak корней. Проще говоря, существует четкое разделение функций различных уровней абстракции. 

В компьютере транзисторы используются как в схеме ЦП, так и жесткого диска.\linebreak Аналогично этому большое число "унифицированных элементов" имеется во всех частях\linebreak растения. Так Создатель достигал экономии средств выражения. Например, клетки служат\linebreak основными строительными блоками всех структур растения; корни, стебли и листья растения\linebreak состоят из клеток. И хотя любой из этих исходных элементов действительно является клеткой,\linebreak существует огромное количество разнообразных клеток. Есть клетки, содержащие и не\linebreak содержащие хлоропласт, клетки с оболочкой, проницаемой и непроницаемой для воды, и даже\linebreak живые и умершие клетки. 

При изучении морфологии растения мы не выделяем в нем отдельные части, отвечающие\linebreak за отдельные фазы единого процесса, например, фотосинтеза. Фактически не существует\linebreak централизованных частей, которые непосредственно координируют деятельность более низких\linebreak уровней. Вместо этого мы находим отдельные части, которые действуют как независимые\linebreak посредники, каждый из которых ведет себя достаточно сложно и при этом согласованно с более\linebreak высокими уровнями. Только благодаря совместным действиям большого числа посредников\linebreak образуется более высокий уровень функционирования растения. Наука о сложности называет это\linebreak возникающим поведением. Поведение целого сложнее, чем поведение суммы его составляющих [6]. 

Обратимся к зоологии. Многоклеточные животные, как и растения, имеют иерархическую\linebreak структуру: клетки формируют ткани, ткани работают вместе как органы, группы органов\linebreak определяют систему (например, пищеварительную) и так далее. Мы снова вынуждены отметить\linebreak присущую Создателю экономность выражения: основной строительный блок всех растений и\linebreak животных - клетка. Естественно, между клетками растений и животных существуют различия.\linebreak Клетки растения, например, заключены в жесткую целлюлозную оболочку в отличие от клеток\linebreak животных. Но, несмотря на эти различия, обе указанные структуры, несомненно, являются\linebreak клетками. Это пример общности в разных сферах. 

Жизнь растений и животных поддерживает значительное число механизмов надклеточного\linebreak уровня, то есть более высокого уровня абстракции. И растения, и животные используют\linebreak сосудистую систему для транспортировки внутри организма питательных веществ. И у тех, и у\linebreak других может существовать различие полов внутри одного вида.
 
\subsection{Пять признаков сложной системы}
Исходя из такого способа изучения, можно вывести пять общих признаков любой сложной системы. Основываясь на работе Саймона и Эндо, Куртуа предлагает следующее наблюдение [7]: 
1. "Сложные системы часто являются иерархическими и состоят из взаимозависимых подсистем, которые в свою очередь также могут быть разделены на подсистемы, и т.д., вплоть до самого низкого уровням." 
2. Выбор, какие компоненты в данной системе считаются элементарными, относительно произволен и в большой степени оставляется на усмотрение исследователя. 
3. "Внутрикомпонентная связь обычно сильнее, чем связь между компонентами. Это обстоятельство позволяет отделять "высокочастотные" взаимодействия внутри компонентов от "низкочастотной" динамики взаимодействия между компонентами" [10]. 
4. "Иерархические системы обычно состоят из немногих типов подсистем, по-разному скомбинированных и организованных" [11]. 
5. "Любая работающая сложная система является результатом развития работавшей более простой системы... Сложная система, спроектированная "с нуля", никогда не заработает. Следует начинать с работающей простой системы". 

Список литературы
[1] Brooks, F. April 1987. No Silver Bullet: Essence and Accidents of Software Engineering. IEEE Computer vol.20(4), p.12. 
[2] Peters, L. 1981. Software Design. New York, NY: Yourdon Press, p.22. 
[3] Brooks. No Silver Bullet, p.11. 
[4] Parnas, D. July 1985. Software Aspects ofStrategic Defense Systems. Victoria, Canada: University of Victoria. Report DCS-47-IR. 
[5] Peter, L. 1986. The Peter Pyramid. New York, NY: William Morrow, p.153. 
[6] Waldrop, M. 1992. Complexity: The Emerging Science at the Edge of Order and Chaos. New-York, NY: Simon and Schuster. 
[7] Courtois, P. June 1985. On Time and Space Decomposition of Complex Structures. Communications of the ACM vol.28(6), p.596. 
[8] Simon, H. 1982. The Sciences of the Artificial. Cambridge, MA: The MIT Press, p.218. 
[9] Rechtin, E. October 1992. The Art of Systems Architecting. IEEE Spectrum, vol.29( 10), p.66. 
[10] Simon. Sciences, p.217. 
[11] Ibid, р. 221. 

\end{document}